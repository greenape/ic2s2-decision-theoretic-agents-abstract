\documentclass[twocolumn]{article}
\usepackage{authblk}
\usepackage[cm]{fullpage}
\usepackage{comment}
\usepackage[square,numbers]{natbib}
\usepackage{graphicx}
\usepackage[subrefformat=parens,labelformat=parens]{subfig}
\usepackage{acronym}
\usepackage{hyperref}
\title{Decision Theoretic Agent Based Modelling: Pregnancy and Alcohol Misuse}
\author{Jonathan Gray\footnote{Corresponding author}}
\author{Jakub Bijak}
\author{Seth Bullock}
\affil{University of Southampton}
\date{\vspace{-5ex}}
\begin{document}
\maketitle
\acrodef{ABM}{Agent Based Model}
\acrodef{CPT}{Cumulative Prospect Theory}

%Let's say about 750 words. That's 250 on what, 250 on why, 200 on what we found. 50 on what next?

\section{Introduction} %(half page)

Agent Based Modelling, and computational approaches more generally, have been advocated for within the social sciences by many authors (naming only a handful: \citet{Axelrod1997, epstein1994growing, agent_zero, gilbert1999simulation, Macy2002a, Resnick,Silverman2011,Silverman2013}). Despite this, there is a pervasive unease about the use of ABM \citep{Waldherr2013}, which we suggest arises in part from weak theoretical underpinnings. The moving parts of ABMs are often remarkably ad hoc, and poorly specified, based on intuitions about individual behaviour even where the model itself is designed as an expression of a well defined theoretical model. To some extent this may seem inescapable, given that there is no overarching and universal theory of human behaviour. We argue that employing a decision theoretic approach can significantly strengthen the psychological plausibility of ABM, and further that grounding agent behaviour in theories provides a valuable opportunity to strengthen and refine the theoretical base.


In this work, we first outline an approach to decision theoretic agents, which draws on the relationship between ABM and game theory (section \ref{dec_theory}), followed by a case study exploring the decision processes involved in disclosure of drinking behaviour during pregnancy under four decision models. Section \ref{results} contrasts the decision models in their ability to capture stylised facts reported in the midwifery literature, which we expand on in section \ref{conc} in discussing the utility of the approach, as well as the limitations it may impose.

\section{Decision Theoretic Agents}\label{dec_theory} %(page)

Theories of decision making can be classified into two broad groups - descriptive, and normative, mirroring the positive vs. normative dichotomy found in economics. Descriptive theories attempt to reflect the process and outcomes of human decision making, including any systematic deviations from rationality. By contrast, normative theories give rational solutions to decision problems. Both classes of theory are potentially of interest here: a descriptive approach might involve specifying agent rules motivated by empirical evidence of how people behave when presented with a particular decision problem. Alternatively, a normative approach could involve using decision theory to derive maximising behaviour in the same context.
Moving between these approaches is convenient, because individual models of decision making are to a large extent modular by virtue of requiring the same inputs to deliver a decision. 

The case for decision theory as an underpinning for agent based modelling rests on several attractive features, key amongst which is the potential to offer psychological plausibility. Recent work in neuroeconomics (see particularly \citep{Padoa-Schioppa2006}, and \citep{Rustichini2009} for a more general review) suggests that the concept of a universal currency for comparing the desirability of outcomes in the brain has a solid evidence base.  
From a pragmatic perspective, the decision rules are not computationally demanding, and are as modular, demonstrated in this case study by the use of multiple theories with no change to the surrounding model.  In a similar vein, altering the decision problem does not require a reconstruction of the agents. This raises the possibility of effectively training a population of agents in one setting, and subsequently allowing them to use the resultant beliefs in different circumstances. Also appealing is the possibility to contribute to the field of decision theory by providing a convenient test for decision rules, since the validity of the large scale behaviour of the model can be seen as indicative of the capability of the decision rule at the local level.

Combining this with a game theoretic framing of the problem gives a well understood framework for formalisation, which can be trivially translated into a set of decision problems by treating games `as if'\footnote{i.e. from the perspective of the agent, since nature is generally disinterested which is not the case here.} against nature \citep{RiosInsua2009}. This recasts the problem from one of reasoning about the psychology of an opponent, to inferring the probability distribution underlying the behaviour of a black box.

\section{The Disclosure Game}\label{case} %(half page)

%The structure of the game. The partial common interest nature of it. The two equilibriums. Generality \& this case study.

In our model, we let the world be populated by two groups of agent: pregnant women, and midwives. Each agent may be one of three types, which for women represents a level of alcohol consumption, and for midwives the extent to which they react negatively to pregnant women drinking. The type of an agent is assigned `by nature', and is initially private information. The two kinds of agent meet, and play a variation on a signalling game \citep{Kreps1987} where the set of possible rewards depends on the type of both players. In a round of the game, the woman sends a signal indicating she is one of the three drinking types, and the midwife may choose to refer them to a specialist\footnote{In the NHS, this is generally a specialist midwife, but may also involve a multi-agency team. See \citet{NICE2010} for the full guidelines.}. Women receive a social payoff contingent on the signal they sent, and the type of the midwife which is revealed to the woman at the end of the round; conversely, the midwife incurs a cost if a referral is made. Both players also receive an additional health outcome payoff depending on the drinking type of the woman, and the referral decision. Hence the woman's payoff is social cost + health outcome, and the midwife's is referral cost + health outcome.

If a woman is referred then that player's game ends (see \citep{NICE2010}), and the referring midwife is always informed of the true drinking pattern. If they are not, then the midwife does not learn the outcome of the game, and the woman may continue to play until they are referred or their baby is born. In the simulation, the size of both populations remains constant, but the individual members of the population of women are replaced over time as they are referred or give birth.

Both groups of agents have a partially common interest in the health outcome, but potentially competing interests in that women would ideally avoid disclosing accurate information to avoid a social penalty arising from the negative reactions of midwives. Clearly this is a somewhat contrived scenario, which neglects much of the nuance and complexity of reality - for example we do not attempt to infer plausible payoffs, or distribution of drinking patterns, nor consider heterogeneity of utility - and is primarily illustrative of the methodology. 

Four decision rules were considered: (1) simple Bayesian risk minimisation, which attempts to infer the distribution of player types, and likelihood of referral; (2) \ac{CPT} \citep{Tversky1992} with Bayesian updating, as a well known descriptive theory; (3) a simple lexicographic rule, which uses one reason decision making along the lines of a Fast and Frugal Heuristic \citep{Gigerenzer1996} and considers only the total payoff from a round; and finally (4) a second Bayesian risk minimisation rule which uses equivalent information to the heuristic method in (3).

\section{Selected Results}\label{results} %(page)


\section{Conclusions and Present Work}\label{conc} %(half page)


\bibliographystyle{plainnat}
\bibliography{library}
\end{document}